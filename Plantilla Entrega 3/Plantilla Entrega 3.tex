\documentclass[letter, 10pt]{article}

%Multicolumna
\usepackage{multirow}

%Pseudo codigo
\usepackage{algpseudocode}

%Cambiado por problemas con tildes
\usepackage[utf8]{inputenc}
\usepackage[spanish]{babel}
\usepackage{amsfonts}
\usepackage{amsmath}
\usepackage{amssymb}
\DeclareMathOperator*{\argmin}{arg\,min}
\usepackage{float}
%Cambiado por problemas con tikz
\usepackage{ifpdf}
\ifpdf
    \usepackage{graphicx}  %If run with `pdflatex`
\else
    \usepackage[dvips]{graphicx} %If run with `latex`
\fi


\usepackage{url}
\usepackage[top=3cm,bottom=3cm,left=3.5cm,right=3.5cm,footskip=1.5cm,headheight=1.5cm,headsep=.5cm,textheight=3cm]{geometry}

\usepackage{tikz}
\usetikzlibrary{babel,arrows}

\begin{document}
\title{ Taller de Métodos y Modelos Cuantitativos \\ Tercer Informe Proyecto \\ Fast Sum of Gaussians  }
\author{Camilo Valenzuela Carrasco\\ 201173030-5
\and
Camilo Rivas Folch \\ 201160089-4
\and
Cristopher Barraza Vicencio \\201173002-k}
\date{\today}
\maketitle

\section{Definición del problema}

El \textbf{Problema de Evolución de Formas} consiste en un conjunto de N puntos que interactuan entre ellos, de tal forma que van modificando su forma conforme el tiempo avanza. Las ecuaciones que describen el problema son las siguientes:

\begin{align}
    \frac{dx_i(t)}{dt} &= \sum_{j=1}^N K(x_i, x_j)\cdot \alpha_j(t) \label{edo}\\
    \frac{d\alpha_i(t)}{dt} &= -\frac{2}{N} \sum_{j=1}^N \{\alpha_i(t)^T \alpha_j(t) \} \cdot K(x_i, x_j) \cdot (x_j(t) - x_i(t)) \\
    x_s(t_0) &= x_o \\
    x_s(t_f) &= x_f
\end{align}

Con $s = 1..N$, $K(x, y) = e^{-\frac{|| x - y ||^2}{2\sigma^2}}$. Con $x_s(t) = (x_s^x(t), x_s^y(t), x_s^y(t))$ la posición de la partícula $s$ en el tiempo $t$ y $\alpha_s(t)$ el vector momento de la partícula $s$ en el tiempo $t$.

Se tiene un problema de valor de frontera (BVP), donde $x_s(t_0)$ y $x_s(t_f)$ son las posiciones de las partículas en $t = 0$ y en el tiempo final respectivamente. Lo que se busca son los valores de momento inicial $\alpha_s(0)$, de tal forma que las partículas pasen de su posición inicial a la final.


\newpage
\section{Dificultades del problema}

\subsection{Suma de Gaussianas}
En las ecuaciones del problema se puede notar que existe una sumatoria con respecto a N puntos.

\begin{equation*}
\frac{d x_i(t)}{dt} = \displaystyle\sum_{j=1}^N e^{-|| x_j(t) - x_i(t)||^2 / \sigma^2} \alpha_j(t)
\end{equation*}
 
El cálculo de esta sumatoria representa una tarea costosa, ya que el número de puntos que se utiliza es grande, por cada punto se tiene que calcular una exponencial.

Si tenemos M puntos donde se tiene que evaluar la sumatoria, y cada sumatoria tiene N terminos, necesitamos realizar $O(MN)$ evaluaciones. Mientras mayor sea la cantidad de puntos, será mucho mayor el tiempo de computación.

Como también tenemos que calcular la posición de los puntos en cada tiempo $t$, mientras más rápido se realice la suma, menos costoso será resolver el BVP.
%\subsection{Seguimiento}

%Otro problema que se posee, es que se quiere hacer un seguimiento de cada %punto, esto es, poder ver la posición inicial de un punto dado y cuál es su %posición final. Esto implica graficar más cosas que los puntos, aumentando %la carga computacional. 

\subsection{Ecuaciones no lineales}

Es posible describir el problema de manera genérica como:

\begin{align}
\begin{pmatrix}
 \frac{d\mathbf{x}}{dt} \\
 \frac{d\mathbf{\alpha}}{dt}
\end{pmatrix}
& = 
\begin{pmatrix}
f_1(\mathbf{x}, \alpha) \\
f_2(\mathbf{x}, \alpha)
\end{pmatrix}
= \mathbf{F}(\mathbf{x}, \alpha) \label{edof}
\end{align}

 El cuál es un problema autónomo y acoplado, dado que para ambas componentes hay una dependencia de ambas variables. La dificultad que existe aquí es que $\mathbf{F}(\mathbf{x}, \alpha)$ es no-lineal, por lo tanto no es simple desacoplar el sistema para poder resolverlo con algún método numérico y  tiene que ser estudiado con detalle. 
 
\subsection{Método de Resolución}

Dado el problema en \ref{edo}, hay que definir una estrategia para resolver el problema que cumpla con ciertas exigencias. Se requiere que el método que se utilice mantenga constante la suma de momentos conforme avance el tiempo, ya que si se utilizan métodos que no conservan bien el momento, como euler, el valor de $\alpha(t_0)$ puede que sea mucho mayor al real.

Se nos ha sugerido el método del trapecio para mantener la estabilidad:

\begin{equation}
z_{n+1} = z_n + (t_{i+1} - t_i) \cdot \mathbf{F}\left(\frac{z_{n+1} + z_n}{2}\right)
\end{equation}

La utilización de un método implícito para mantener la estabilidad del problema requiere un esfuerzo adicional, sobre todo teniendo en cuenta la problemática planteada por la función utilizada en \ref{edof}, donde no es fácil resolver para $z_{n+1}$ dada la naturaleza explicada anteriormente.

Cabe notar, que para poder solucionarlo se requiere conocer el jacobiano de la función F, debido a que por ser un método implícito, es necesario resolverlo con el método de newton. No obstante, este jacobiano no es simple de calcular y agregaría mucho más esfuerzo computacional al problema.

Cabe destacar que el problema en sí está formulado para que el momentum del sistema se mantenga constante en el tiempo, por lo tanto, es posible usar una función más exacta, como Odeint de Scipy, que además se encarga de aproximar el jacobiano del sistema, por lo que se decide utilizar éste método por el momento.


\subsection{Implementación}
Una última dificultad está en la forma de implementación. Al momento de implementar las sumas rápidas para resolver el problema, se tiene que pensar en generalizar lo más posible, para poder agregar 

ya que para asegurar mayor rendimiento posible en cuanto a tiempo, lo ideal sería utilizar programación paralela o algún método pueda acelerar los cálculos computacionales, como sería el uso de las GPU, en específico programando el algoritmo en CUDA. Esta dificultad está en cuanto a que la forma


\section{Trabajo realizado}
\subsection{Sumas Rápidas}
En lo que respecta a las sumas rápidas, hasta el momento se ha implementado una versión de sumas rápidas en una dimensión utilizando \textit{Fast Multiplole Method (FMM)}, en la cual se utiliza la división de puntos en regiones cercanas y regiones lejanas, pero no se ha incluido aún la división del espacio en Boxes, que agrupan puntos cercanos entre si, y dependiendo del punto que se está evaluando, la Box se puede comportar como un punto de la región lejana o varios puntos de la región cercana.


Este trabajo se hizo para entender la base de las sumas rápidas, y comenzar a comprender cómo hacen para aproximar los valores reales en menor tiempo de ejecución.

Para esto se utilizó la Ley de Gravitación Universal de Newton, que al igual que el problema que estamos estudiando es un N-Body problem, donde cada partícula interactua con todas las otras en algún modo, en el caso de la ley de gravitación la interacción depende de la distancia de la partícula estudiada con relación a las demás.

La interacción de una carga puntual $z_0$ está dada por la suma de todas las interacciónes con las otras cargas puntuales $z$ con carga $q$.
\begin{equation}
\phi_{z_0}(Z) = q \log(z-z_0) \label{charges}
\end{equation}          

Para utilizar FMM, primero se define un radio $r$ , para dividir el dominio en una región cercana y una lejana. En la región cercana definida como los puntos $z \in [z_0 - r , z_0+ r]$ se evalua la interacción directamente con \ref{charges}. 

Para la región lejana se utiliza \textit{Multipole Expansion} que transforma la ecuación \ref{charges} en una serie infinita, y como resultado tenemos la siguiente ecuación:
\begin{equation}
\phi(z) = Q \log(z) + \sum_{k=1}^\infty \frac{a_k}{z^k}
\end{equation}

donde 
\begin{align*}
    &Q = \sum_{i = 1}^m q_i    &a_k = \frac{\sum_{i = 1}^n -q_i z_i^k}{k}
\end{align*}
Dependiendo de la precisión deseada la serie puede ser truncada a $p$ términos, además como $a_k$ no depende del punto $z$ donde se está evaluando $\phi$, entonces puede ser precalculado para cada $k \in [0,p]$.

Al sólo tener que calcular $p$ términos de la serie y poder precalcular los valores de $a_k$ se puede acelerar el tiempo de ejecución del algoritmo, en esto se basa el FMM.

Tomando en cuenta lo aprendido con el ejemplo en 1D, pasamos a implementar el algoritmo Fast Gauss Transform, que busca realizar de manera eficiente la siguiente sumatoria:
\begin{equation}
G(t_i) = \sum_{j=1}^N q_j e^{\frac{-|t_i - s_j|^2}{\sigma}}
\end{equation}
donde $t_i$ son los puntos Targets y $s_j$ son los puntos Sources, y $q_j$ el peso o carga del source. Esta ecuación es similar a la de nuestro problema, pero nuestro $q_j$ seria el vector de momentos de la partícula $j$, $\alpha_j(t)$ o se tendrían que pre-calcular los $\{\alpha_j(t)^T \alpha_i(t) \} \cdot (x_j(t) - x_i(t)) $ y utilizar esos valores para los $q_j(t)$.

\subsubsection{Algoritmo Fast Gauss Transform}
Para comenzar, se tiene como supuesto que el dominio donde están los sources y targets están pertenecen a $t_i,s_j \in B_0 = [0,1]^d$, con d la cantidad de dimensiones de los sources y targets.

El algoritmo divide el dominio $B_0$ en Boxes de tamaño $r\sqrt{\delta}$, con un total de $Nside^d$ Boxes, el paper dice que para la elección de $Nside$ hay que buscar el valor de $r \leq \frac{1}{2}$ de tal forma que $Nside$ de un valor entero.

Para calcular $Nside$, nos centraremos solo en una dimensión, como el dominio de cada dimensión va de [0,1], y cada Box mide $r\sqrt{\delta}$, por lo que 
$$Nside = \lceil{\frac{max\_dom(B_0) - min\_dom(B_0)}{r\sqrt{\delta}}\rceil} = \lceil{\frac{1}{r\sqrt{\delta}}\rceil}$$

Como aproximamos $Nside$ al entero más cercano, calculamos el nuevo r
$$ r' = \frac{1}{Nside \sqrt{\delta}}$$

Ahora ya sabemos cuantas Boxes va a tener nuestro algoritmo, vamos a tener dos tipos de Boxes, TargetBoxes (B) son las que acumularan el potencial de cada partícula y  las SourceBoxes (C) que interactuan con las TargetBoxes. 

Utilizando r' y la precisión $\epsilon$ calculamos p, que va a ser la variable utilizada para truncar las series infinitas, y para elegir que método de aproximación utilizar, además calculamos $n$ que es la cantidad de Boxes que pueden interactuar con una TargetBox y depende de los parámetros $\delta$ y $\epsilon$.

Para la aproximación se utilizan 4 métodos, todo depende de la cantidad de términos que hayan en las Boxes que se están analizando, sea $M_C$ la cantidad de target en una TargetBox C y $N_B$ la cantidad de sources de una SourceBox B. Primero se eligen puntos un punto de corte para cada caja, $N_F$ para las SourceBox y $M_L$ para las TargetBox, que en nuestro caso usaremos $p^d$ en ambas, ya que esa es la cantidad de términos en la que se truncan las series utilizadas. Luego:
\begin{itemize}
\item Si $N_B < N_F$: No hay suficientes términos para truncar la serie, por lo que la SourceBox B envia $N_B$ Gausseanas a la interacción. (Se calcula directo).
\item Si $N_B >= N_F$: Hay términos suficientes para truncar la serie, por lo que B en vez de enviar Gausseanas envia una \textbf{Hermite Expansion} a la interacción.

\item Si $M_C < M_L$: C evalua directamente lo que le envia B (Sea $N_B$ Gausseanas o la Hermite Expansion)

\item Si $M_C >= M_L$: C acumula lo que le envía B mediante una serie de Taylor.
\end{itemize}


La \textbf{Hermite Expansion} utiliza la \textit{Hermite Function} h(t) y está centrada en el SourceBox con centro $s_B$ y es de la forma:
\begin{equation}
    G(t) = \sum_{\alpha \geq 0} A_\alpha h_\alpha\left( \frac{t- s_B}{\sqrt{\delta}}\right)
\end{equation}
donde los coeficientes $A_\alpha$ están dados por:
\begin{equation}
 A_\alpha = \frac{1}{\alpha} \sum_{j=1}^{N_B} q_j \left(\frac{s_j - s_b}{\sqrt{\delta}}\right)^\alpha
\end{equation}
En nuestro caso $\alpha$ va a ser un indice múltiple que depende de la cantidad de dimensiones del problema, por ejemplo para puntos en un espacio de 3 dimensiones (x,y,z), $\alpha$ va a ser una 3-tupla.

En el caso de que la TargetBox C tenga que acumular la interacción como una serie de Taylor, tiene que utilizar las siguientes ecuaciones:
\begin{itemize}
\item Si B nos envia las $N_B$ Gausseanas, C tiene que acumular las interacciones utilizando:
\begin{equation}
 G(t) = \sum_{\beta \geq 0} B_\beta \left( \frac{t-t_C}{\sqrt{\delta}} \right) ^\beta
\end{equation}
Con los coeficientes:
\begin{equation}
B_\beta = q \frac{(-1)^|\beta|}{\beta!}h_b\left(\frac{s_j - t_C}{\sqrt{\delta}}\right)
\end{equation}

\item En el caso de que B nos envié una Hermite Expansion, C tiene que acumular las interacciones utilizando:
\begin{equation}
G(t) = \sum_{\beta \geq 0} C_\beta \left(\frac{t-t_C}{\sqrt{\delta}}\right)^\beta
\end{equation}
con los coeficientes $C_\beta$ se calculan con:
\begin{equation}
C_\beta = \frac{(-1)^{|\beta|}}{\beta!} \sum_{a\leq p} A_\alpha h_{\alpha + \beta} \left( \frac{s_B - t_C}{\sqrt{\delta}} \right)
\end{equation}
\end{itemize}


Utilizando estas aproximaciones pasamos de tener una complejidad O(MN) a O(M+N), por lo que el tiempo de ejecución baja considerablemente.
%También, contamos con código funcional de Fast Gaus Transform para 2 dimensiones, dónde quedó para nosotros estudiar cómo funciona, es decir, parámetros que recibe, transformaciones que hace, etc. Se puede encontrar más detalle en las referencias.


\subsection{Resolviendo el Problema de Evolución de Formas}
En la formulación del problema a resolver, se ha comenzado a avanzar en el planteo de este para tratar de llegar a una representación que permita utilizar alguna técnica conocida sobre el conjunto de datos para lograr resolver el problema. Para esto se buscó alguna forma de desacoplar el sistema, pero como ámbas ecuaciones dependen de las mismas variables y debido a su naturaleza no lineal, no se encontró una forma sencilla de desacoplarlo.

Por otra parte, para resolver el BVP, se nos recomendó utilizar \textbf{Shooting Method}, para transformar la Ecuación Diferencial a un problema de valor inicial (IVP), y luego utilizar Backward Euler con la regla del punto medio, ya que es mucho mejor para mantener el momento del sistema.

% BEGIN APORTE CRIS
% SE AGREGÓ DESPUÉS, POR SI HAREMOS LATEX DIFF
\subsubsection{Shooting Method}
Las ecuaciones diferenciales que componen este problema y los datos con los que se cuentan componen un BVP, el cual
puede ser resuelto a través de \textit{Shooting Method}, convirtiendo el problema en un IVP, en el cual será necesario
justamente encontrar el impulso inicial que llevó el sistema desde el estado inicial al estado final. Por lo tanto, 
resolver el problema mediante \textit{Shooting Method} se enfocará en resolver:

\begin{equation}
F(\alpha) = X(1) - \hat{X}_\alpha(1)
\end{equation}

Siendo \(\alpha\) el impulso inicial, \(X(1)\) las posiciones de cada uno de los puntos del sistema en el tiempo \(t = 1\) y
\(\hat{X}_\alpha(1)\) la posición de los punto en el tiempo \(t = 1\) dado un cierto valor de \(\alpha\). El objetivo el 
lograr minimizar el valor de \(F(\alpha)\).
% END APORTE CRIS
\subsubsection{Utilizando Sumas Rápidas}
Para realizar la integración sobre la variable temporal, necesitamos calcular el cambio de la posición y momentum de la partícula, para esto tenemos que calcular la ecuación (1) y (2), utilizando sumas rápidas.

La primera ecuación se puede calcular utilizando la forma estándar del Fast Gauss Transform que se puede observar en la ecuación (9).

En cambio la segunda ecuación se vuelve más compleja para evaluarla, porque además de tener el kernel gausseano se tiene un producto punto de vectores y la diferencia entre el punto target y el source.

Para poder realizar esta suma eficientemente hay que manejarla hasta tener algo parecido a la ecuación (9).

Como se está trabajando en 3 dimensiones el producto punto de $\alpha_i(t)^T \alpha_j(t)$ se puede expresar de la siguiente forma

\begin{equation}
\begin{bmatrix}
\alpha_i^x\\
\alpha_i^y\\
\alpha_i^z\\
\end{bmatrix} \cdot
\begin{bmatrix}
\alpha_j^x ,
\alpha_j^y ,
\alpha_j^z ,
\end{bmatrix} =
\alpha_i^x \cdot \alpha_j^x +
\alpha_i^y \cdot \alpha_j^y +
\alpha_i^z \cdot \alpha_j^z 
\end{equation}

Por lo que la segunda ecuación va quedando de la siguiente forma:
\begin{equation}
\frac{d\alpha_i(t)}{dt} = -\frac{2}{N} \sum_{j=1}^N \left( \alpha_i^x \cdot \alpha_j^x +
\alpha_i^y \cdot \alpha_j^y +
\alpha_i^z \cdot \alpha_j^z \right) \cdot K(x_i, x_j) \cdot (x_j(t) - x_i(t))
\end{equation}
Luego podemos multiplicar el kernel por cada término
\begin{equation}
\frac{d\alpha_i(t)}{dt} = -\frac{2}{N} \sum_{j=1}^N \left( \alpha_i^x \cdot \alpha_j^x K(x_i, x_j) +
\alpha_i^y \cdot \alpha_j^y K(x_i, x_j) +
\alpha_i^z \cdot \alpha_j^z K(x_i, x_j) \right) \cdot (x_j(t) - x_i(t))
\end{equation}
Ahora multiplicamos el termino de la izquierda por cada término de la derecha

\begin{align*}
    \frac{d\alpha_i(t)}{dt} =& -\frac{2}{N} [ \sum_{j=1}^N \left( \alpha_i^x \cdot \alpha_j^x K(x_i, x_j) x_j(t) +\alpha_i^y \cdot \alpha_j^y K(x_i, x_j) x_j(t) +\alpha_i^z \cdot \alpha_j^z K(x_i, x_j) x_j(t)  \right)
    \\ &-\sum_{j=1}^N \left( \alpha_i^x \cdot \alpha_j^x K(x_i, x_j) x_i(t) +\alpha_i^y \cdot \alpha_j^y K(x_i, x_j) x_i(t) +\alpha_i^z \cdot \alpha_j^z K(x_i, x_j) x_i(t)  \right) ]
\end{align*}

Ingresamos la sumatoria a cáda término, dejando una suma de varias sumatorias pequeñas, dejamos fuera de cada sumatoria las variables que no dependen de ella

\begin{align*}
    \frac{d\alpha_i(t)}{dt} =& -\frac{2}{N} \cdot [
    \alpha_i^x \sum_{j=1}^N  \alpha_j^x x_j(t) K(x_i, x_j) 
    +\alpha_i^y \sum_{j=1}^N \alpha_j^y x_j(t) K(x_i, x_j) 
    +\alpha_i^z \sum_{j=1}^N \alpha_j^z x_j(t) K(x_i, x_j)  \\ 
    &- x_i(t)  (\alpha_i^x \sum_{j=1}^N \alpha_j^x K(x_i, x_j)
    +\alpha_i^y \sum_{j=1}^N \alpha_j^y K(x_i, x_j) 
    +\alpha_i^z \sum_{j=1}^N \alpha_j^z K(x_i, x_j)  
   )
    ]
\end{align*}

Ahora tenemos nuestra segunda ecuación con seis sumatorias, que pueden ser evaluadas utilizando Fast Gauss Transform, las primeras tres sumatorias, los pesos que se utilizan es la multiplicación de un componente del momentum ($\alpha_j^d$) con el vector de posición ($x_j$), las otras tres sumatorias son más parecidas a la ecuación numero (1).

Para simplificar un poco la evaluación, realizamos cada dimensión por separado, por lo que para la primera ecuación se calculan tres sumas rápidas, una por cada dimensión, y para la segunda se calculan dieciocho sumas rápidas.

Como nuestra implementación de Fast Gauss Transform utilizando Python se volvió tediosa (por la complejidad de calcular los coeficientes hermitianos utilizando recursión), además la implementación incompleta se demoraba más en computar que la suma directa, por lo que se decidió utilizar una biblioteca ya existente, llamada Figtree, la que utiliza IFGT (Improved Fast Gauss Transform) y Figtree, para aproximar las sumas.

Las primeras pruebas realizadas, con los datos entregados por Felipe Arrate, que son 4028 puntos en 3 dimensiones, nos entregaron los siguientes resultados.

\begin{center}
    \begin{tabular}{c|c|c|c}
                     & Suma Directa & Figtree & Diferencia  \\ \hline
    Primera Ecuación & 3.79s & 170ms & $4.24e^{-13}$ \\
    Segunda Ecuación & 13.6s & 621ms & $2.48e^{-14}$ 
    \end{tabular}
\end{center}
\section{Plan de Acción}

Dado el trabajo realizado, se propone el siguiente plan de acción:

\begin{itemize}
\item Comprender como funciona el algoritmo Improved Fast Gauss Transform, al transformar la suma de gausseanas a una serie de Taylor.
\item Unir los métodos utilizados para la optimización en el cálculo de las derivadas con el método elegido
para resolver el problema.
\item Realizar pruebas para verificar que las técnicas utilizadas para optimizar el cálculo y el método
para resolver el problema funcionan correctamente y en un tiempo de ejecución reducido, además de verificar
que al resolver el problema se esté conservado el momentum del sistema.
\item Luego de verificar que todo funciona correctamente, se pasará a representar graficamente el comportamiento
que tienen los puntos desde la situación inicial a la final.
\end{itemize}


\section{Implementaciones de Fast Gauss Transform o Similares}

\begin{itemize}
\item Fast computation of Gauss transforms in multiple dimensions \\ \url{https://github.com/vmorariu/figtree}
\item Fast Gauss Transform  \\ \url{https://github.com/gadomski/fgt}
\item Kernel density estimation which was optimized by the fast Gauss transform. \\ \url{https://github.com/y-mitsui/fastKDE}

\item Fast Gauss Transform (Python) \\ \url{https://github.com/josephpmcdonald/FastGauss}
\end{itemize}



%\section{Bibliograf\'ia}
\bibliographystyle{apalike}
\end{document} 
